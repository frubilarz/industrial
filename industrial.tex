% Template:     Informe/Reporte LaTeX
% Documento:    Archivo de ejemplo
% Versión:      5.5.4 (16/09/2018)
% Codificación: UTF-8
%
% Autor: Pablo Pizarro R. @ppizarror
%        Facultad de Ciencias Físicas y Matemáticas
%        Universidad de Chile
%        pablo.pizarro@ing.uchile.cl, ppizarror.com
%
% Manual template: [http://latex.ppizarror.com/Template-Informe/]
% Licencia MIT:    [https://opensource.org/licenses/MIT/]

\section{Introducción}
Día a día el parque automotriz aumenta y es necesario mantener y tener nuevas señaléticas. Debido a esta necesidad es que nace Serviplott, una empresa la cual se encarga de vender servicios a distintos municipios los cuales requieren nuevas señaléticas y mantención de las señales viales ya existentes para así mejorar la experiencia de conducción.
\vspace{5mm}

Toda municipalidad tiene como tarea manejar las señales viales existentes en su sector. Al tener un número alto de señales en los distintos sectores es que las municipalidades optan por externalizar este servicio y se lo dejan a empresas importantes tales como Serviplott.
\vspace{5mm}

En el presente trabajo se desarrollará un completo análisis a la gestión de los procesos Industriales de Serviplott, identificando todos los problemas asociados a esta área, para luego proponer un plan de mejora en el ámbito de la Gestión de Procesos.
\vspace{5mm}

\insertimage[]{serviplott/logo2}{width=10cm}{Logo Empresa.}\newpage
\section{Objetivos}
	\subsection{Objetivos Generales}
Analizar un proceso industrial de la empresa “Serviplott Servicios Gráficos y Señalización Vial Ltda”, y proponer planes de mejora.
	\subsection{Objetivos Específicos}
	\begin{itemize}

		\item Realizar un relevamiento de información de la empresa seleccionada para 					diagnosticar el estado actual respecto a la organización, sus procesos 				y sistemas.
		\item Determinar las soluciones a las oportunidades de mejora detectadas.
		\item Formular las recomendaciones necesarias para implantar los proyectos y
		acciones de mejora.
		\item Realizar una propuesta de mejoramiento, que refleje las oportunidades de 		mejora posibles para la empresa.
	\end{itemize}
	\newpage
\section{Alcances Y Limitaciones}
	\subsection{Alcances}
	A partir de los antecedentes recolectados de “Serviplott”, se realizará un estudio de un proceso industrial de la organización que sea relevante de acuerdo a la información respecto a la organización,  sus procesos y sistemas, el cual contempla los aspectos, tales como la situación actual de la empresa, el soporte tecnológico, así como también las mejores prácticas, entre otros. Con toda esta información, en conjunto con las estrategias de la empresa, se definirá una lista de oportunidades de mejora, procesos, además de proponer planes de mejora para ese proceso en específico. 
	
	\subsection{Limitaciones}
	Existirá limitación en cuanto a la información a la que pudiese ser accedida desde el punto de vista financiero y/o estratégico por motivos de la empresa y de sus clientes, en este caso distintas municipalidades, por lo que algunas estrategias de mejora podrían no ser del todo óptimas. Es importante mencionar que el Plan de Mejoramiento entregado será considerado como una propuesta.
\newpage
\section{Antecedentes Generales de la Empresa}
	\subsection{Identificación de la Empresa}
	Fundada en 2002, Serviplott es una empresa conformada por profesionales del área de la ingeniería mecánica y la ingeniería en transporte. Estas características le dan a la empresa flexibilidad, rapidez y soporte técnico que demanda el mercado de la infraestructura vial. En Serviplott creen en una relación personalizada con sus clientes y por ello ofrecen tiempos de respuesta de acuerdo a sus particulares necesidades con altos estándares de calidad.
\vspace{5mm}

Inicialmente se encargaban de generar gráficos publicitarias y seguridad, también entregando proyectos de ingeniería en tránsito. Con el tiempo serviplott comenzó a especializarse solamente en soluciones de ingeniería en tránsito, donde se especializó en señales viales.\vspace{5mm}

A partir de ello, serviplott comienza a licitar en las distintas comunas para así ser una empresa de renombre en mantención e instalación de señaléticas viales. Hoy en día serviplott se encuentra encargado de comunas tales como Santiago, Lo Barnechea, Macul, Ñuñoa.\vspace{5mm}

Serviplott como empresa se encuentra en una constante actualización de técnicas y tecnologías capacitando a sus distintos trabajadores  en las distintas áreas. 


	\subsection{Misión}
	Serviplott es una empresa conformada por profesionales del área de la ingeniería mecánica y la Ingeniería en transportes. Estas características le dan a la empresa la flexibilidad, rapidez y soporte técnico que demanda el mercado de la infraestructura vial. En Servilplott creemos en una relación personalizada con nuestros clientes y por ello ofrecemos tiempos de respuesta de acuerdo a sus particulares necesidades con altos estándares de calidad.
	\subsection{Visión}
Consolidarnos como el mejor proveedor de productos y servicios viales, siendo reconocidos por el mercado como una empresa seria, profesionales y confiable orientada a la satisfacción de sus clientes y de los usuarios del servicio.

	\subsection{Valores}
	Nos importa el bienestar de nuestro entorno y colaboradores, preocupándonos de mantener niveles óptimos de seguridad y cumpliendo con las exigencias de las industrias que atendemos. Los materiales y productos utilizados en nuestras obras provienen de proveedores con trayectoria a nivel mundial. Por otro lado nos aseguramos de entregar un servicio de mantención y post venta con respuestas ágiles y eficaces a nuestros clientes.


Es nuestra responsabilidad cumplir con esta promesa y el mayor reconocimiento está en la decisión de nuestros clientes de preferir contratar nuestros servicios. Somos más que energía térmica: estamos comprometidos con el desarrollo y
trabajo bien hecho.
	\subsection{Organigrama}
	La siguiente figura es una representación gráfica de la empresa, en el cual se muestran las relaciones de jerarquías entre sus diferentes partes. 
	
	
	Este organigrama fue entregado por la empresa que se está analizando, y es un modeleo abstracto y sistemático que permite obtener una idea uniforme y sintética de la estructura formal de la empresa Serviplott.
\insertimage[]{serviplott/organigrama}{width=13cm}{Organigrama Empresa.}
		\subsubsection{Descripción de cargos}
    \begin{enumerate}
    \item CEO :	Es el director general de la empresa sobre el cual caen todas las responsabilidades.
    \item Gerente General : Es el encargado de coordinar los elementos Finanzas, Producción, Logística y Comercial.
    \item Finanzas
    \begin{enumerate}[a)]
        \item Contabilidad : Esta área es encarga de emitir facturas y llevar registro de todos los procesos contables de la empresa.
        \item Cobranza : Existen cobradores encargados de ir a buscar los pagos y de llamar a clientes con deuda vigente.
    \end{enumerate}
    \item Producción
    \begin{enumerate}[a)]
        \item Técnicos de producción : Empleados encargados de producir los pedidos.
        \item Técnicos en terreno : Personal que acude a terreno para montar el pedido.
    \end{enumerate}
    \item Logística
    \begin{enumerate}[a)]
        \item Bodegueros : Encargados de llevar registro del inventario asi como entradas y salidas de la bodega. También genera una solicitud de materias primas cuando están en un bajo nivel.
        \item Transportistas : Mueven los insumos desde los proveedores hasta la bodega y movilizan a los técnicos en terreno.
    \end{enumerate}
    \item Comercial
    \begin{enumerate}[a)]
        \item Ventas : Personal encargado de gestionar las ventas que se hacen.
        \item Marketing : Área encarga del marketing para promocionar la empresa por distintos medios.\newpage
    \end{enumerate}
    \end{enumerate}
	\subsection{Mix de productos}
	Serviplott Servicios Gráficos y Señalización Vial Ltda es una empresa que ofrece:
	\begin{itemize}
	\item Mantención de Señales viales.
	\item Instalación de Postes.
	\item Instalación de Señal.
	\insertimage[]{serviplott/mixproducto1}{width=8cm}{Señalética.}
	\item Vallas peatonales. 
	\insertimage[]{serviplott/mixproducto2}{width=8cm}{Valla peatonal.}
	\item Demarcación de Reservados. 
	\item Conos de PVC.
	\insertimage[]{serviplott/mixproducto3}{width=8cm}{Cono PVC.}
	\end{itemize}
	\textbf{Mantención de Señales Viales:} \\
\vspace{3mm} Este mix consta de 10 partes que se pueden realizar en una señal ya existente. la cual puede ir desde:
	\begin{itemize}
	\item Enderezar o aplomar poste. 
	\item Instalación de placa Señal. 
	\item Pintado de poste.
	\item Limpieza de señal.
	\item Cambio de Poste.
	\item Retiro solo placa.
	\item Anclaje de poste.
	\item Traslado de Señal completa.
	\item Reinstalación de Señal completa.
	\item Retiro de Señal completa (cabe destacar esta última debido a que en algunas 		comunas y en algunas señales tiene cobro.) 
	\end{itemize}
	
	\textbf{Instalación de poste: }\\
	\begin{itemize}
	\item Poste Omega de 3.0 con base
	\item Poste Omega con extensión
	\item Poste Omega 2.0
	\item Poste Omega
	\item Poste Omega 3.0
	\item Poste Omega 3.5
	\item Poste 50x50 con base
	\item Poste 50x50
	\end{itemize}
	\textbf{Instalación de Señales:}\\
	Para las señales existen distintos tipos de dimensiones y en algunas ocasiones figuras, tal como muestra la siguiente tabla.
	\insertimage[]{serviplott/dimensiones}{width=15cm}{Dimensiones de Señales.}
	\textbf{Vallas peatonales: }\\
	Dentro de este producto se divide en 4 trabajos que se pueden realizar sobre las vallas peatonales:
	\begin{itemize}
	\item Pintura de vallas peatonales.
	\item Reparación de vallas peatonales.
	\item Instalación de vallas peatonales.
	\item Provisión e instalación de vallas peatonales.
	\end{itemize}

\textbf{Demarcación de Reservados: }\\
	La demarcación viene a ser la pintura que se coloca en los distintos espacios para reservados que sean solicitados ya sea por espacios públicos como demarcación de estacionamientos para personas con capacidad reducida, o espacios solicitado de privados a la municipalidades como puede ser el espacio de taxis y colectivos.
    \subsection{Procesos internos}
    La siguiente figura (Procedimiento de ejecucion de proyectos) es un cuadro de los procedimientos que se hacen dentro de la empresa para la ejecución de proyectos.
    \insertimage[]{serviplott/procedimientoejecuciondeproyectos}{width=14cm}{Procedimiento de ejecucion de proyectos}
    
    
    Esta figura referencia todos los procesos de la empresa serviplott en tanto a logística
    \insertimage[]{serviplott/procesologistica}{width=14cm}{Procesos de logística}

\section{Análisis de procesos}
	\subsection{Identificación de inputs}

	El principal input para el proceso de fabricación de las señaléticas viales, son las materias primas necesarias para su fabricación y los aparatos correspondientes para trabajar las materias primas. Las materias primas son entregadas un por un grupo de proveedores regularmente, mientras que las máquinas para operar la materia prima se compran sólo cuando hagan falta.
	
	Entre las materias primas tenemos:
	
    \begin{enumerate}[1)]
    \item Papel de impresion (Vinil autoadhesivo) : El principal componente donde queda la señalética. Es un material de plástico bondadoso y resistente. Pro su flexibilidad se amolda y adhiere fácilmente a cuaqluier superficie (plana o curva). Es ideal para ambientes interiores y exteriores.
    \item Metal pintado y secado al horno : Ideal para señales que se van a colocar en las ciudades ubicadas en altura y con climas inclementes.
    \item Acríclico : Es un material que tiene un brillo y acabado elegante. Es excelente para hoteles, oficinas, restaurantes y lugares exclusivos ya que no altera la estética del lugar por su elegancia.
    \item Energía eléctrica : Factor que interviene en la produción, específicamente dandole energía a las maquínas que imprimen las señales y a las que cortan material.
    \item Poste omega :  poste acero laminado en caliente para sustentanción de señalética vial.
    \item Barras de acero : Principal componente de las vallas peatonales que se hacen a partir de las especificaciones que muestra la figura siguiente.
    \insertimage[]{serviplott/valla1}{width=7cm}{Especificaciones para la valla peatonal.}
    \newpage
    \end{enumerate}
    
    
    Las materias primas tienen el siguiente flujo en la empresa.
        \insertimage[]{serviplott/flujostock}{width=8cm}{Flujo de stock de materias primas en la empresa.}

    
    
	\subsection{Identificación y descripción de procesos}

Dentro de las distintas actividades la empresa divide en tres procesos los cuales son:
\begin{enumerate}[1)]



\item Proceso de Recepción y almacenamiento
\begin{itemize}
\item  La recepción de materiales se define como las actividades que se requieren para recibir, comprobar e informar la llegada de los bienes que se han adquirido o que han llegado para resguardo. Para esto el procedimiento es el siguiente: 

\item El encargado de bodega o quien lo subrogue verifica concordancia de materiales recibidos con solicitud de materiales y/u orden de compra y recepciona conforme guía de despacho o factura  según corresponda. En esta oportunidad se aprovecha de comprobar el buen estado de los insumos o materiales recibidos. 
\item De no haber concordancia o de presentarse insumos y materiales en mal estado, se procede a la no aceptación del pedido, a no ser que se verifique nota de crédito por diferencias. 
\item Es posible hacer recepciones parciales respecto a todo lo consignado en la solicitud de materiales y/u orden de compra. En este caso, se enviará a la Dirección de Adquisiciones la factura correspondiente toda vez que llegue nota de crédito por parte del proveedor, en un máximo de 10 días hábiles. 
\item Se solicita al proveedor factura original y copia color verde. Las facturas deben concordar en cantidad y valores con la orden de compra y estar legibles.
\item Verificado el buen estado de insumos o materiales recepcionados, se procede a la determinación de lugar físico de acuerdo a sección correspondiente. 
\item El almacenamiento de los bienes guardará relación con su rotación y posibilidad de ser maniobrado para efectuar la entrega, siendo los de mayor rotación y los de mayor dificultad de maniobra dejados en lugares próximos a las puertas de entrega.
\item Al almacenar las cajas, la etiqueta o rótulo de identificación debe quedar visible, hacia delante.  
\end{itemize}
  
  
\item Proceso de Producción
\begin{itemize}
\item En la realización de esta clase de proyectos es indispensable observar una estricta planeación para obtener resultados positivos. 
\item Luego de estar el diseño listo de la señalética se manda a producción.
\item La primera etapa es imprimir la señalética.
\item Se construye un poste omega según las dimensiones de la señalética. Esto consiste en cortar los postes omega con las dimensiones acordes para soportar una señal de algún peso específico.
\item Se corta el metal donde se pegará la impresión de la señalética.
\item Para el caso de las vallas peatonales, se soldan bloques de vallas según el largo requerido para colocar en la solera de la calle.
\end{itemize}

\item Proceso de Despacho
\begin{itemize}
\item El despacho consiste en llevar a terreno la señalética o vallas peatonales desarmadas.
\item En el lugar se recibe a los técnicos generalmente un funcionario de la municipalidad de la comuna donde se colocará la valla peatonal o la señalética
\item Los técnicos ensambla la valla peatonal o la señalética en el lugar que se colocará.
\item Una vez finalizado el trabajo de colocación del producto, un funcionario municipal debe dar el visto bueno y firmar una carta de producto completado.
\end{itemize}
\end{enumerate}	
	
	Estos procesos no son aislados, más bien responden a un orden en una cadena de producción la cual se muestra en la siguiente imagen:
	 \insertimage[]{serviplott/procesos}{width=10cm}{Cadena de Producción}

\subsubsection{Proceso de recepción y almacenamiento}
	En el proceso de recepción y almacenaje tiene el siguiente patrón :
	En la figura de abajo 
	\begin{enumerate}[1)]
	\item Llegada del camión a serviplott : Hay un chofer que transporta en un camión hasta las dependencias de serviplott los materiales solicitados previamente. 
    \insertimage[]{serviplott/almacenaje1}{width=8cm}{Llega de camión con materias primas}
    \item Descarga de insumos : Las materias primas son recibidas y descargadas en la bodega para ser almacenados luego de verificar la orden de los productos.
    \insertimage[]{serviplott/almacenaje2}{width=10cm}{Descarga de insumos}
    \item Clasificación de productos : Se clasifican los productos de la orden de compra según el material, para facilitar el pedido de insumos para llevarlos a producción.
    \item Control de inventario : A los insumos recibidos en la bodega se les asigna un código interno el cual será utilizado para realizar los inventarios cada cierto tiempo.
    \end{enumerate}
		\subsubsubsection{Normas de almacenamiento}
		La zona de almacenamiento debe estar perfectamente delimitada y señalizada, no se puede almacenar fuera del lugar designado. Debe evitarse que los embalajes o cualquier tipo de objeto impidan la visualización de señalización y no estén bloqueadas las salidas de evacuación.
	\insertimage[]{serviplott/almacenaje3}{width=10cm}{Normas de seguridad en la bodega}
 
 
	Esto queda más especificado en el siguiente esquema : 
	\insertimage[]{serviplott/almacenaje4}{width=9cm}{Esquema de seguridad.}
	\newpage
		 \subsubsubsection{Señalización en la bodega}
		 Alguna de las señaléticas usadas en el proceso productivo y de almacenaje de serviplott se muestran en las siguientes figuras.
		 	\insertimage[]{serviplott/padrones1}{width=12cm}{Rótulos en bodega y zona de producción.}
		 También existen ciertos rótulos para marcar ruta de evacuación y señales de emergencia.
		 	\insertimage[]{serviplott/padrones2}{width=9cm}{Rutas de emergencia.}
		\subsubsubsection{Instructivos}
		
		Serviplott cuenta con manuales para diversas cosas, como por ejemplo instructivos sobre procesos, administración de bodegas, como también instructivos sobre la instalación y mantención de las diversas maquinarias con las que cuentan. 		 		

	\subsubsection{Proceso de producción}		  
	
En esta actividad se procede a trabajar las materias primas, se confeccionan los productos finales, los cuales seran explicados en \ref{output}

Antes de llevar un pedido de señalética a producción se diseñan los símbolos que contendrán a través de un software como se muestra en la siguiente figura.
\insertimage[]{serviplott/produccion3}{width=9cm}{Diseño en software.}
\vspace{5mm}

Luego de estar listo el diseño se manda a imprimir con impresoras especializadas en impresion de Vinilos adhesivos.
\insertimage[]{serviplott/produccion2}{width=6cm}{Impresión de vinilo adhesivo.}


Luego de tener lista la impresión del pedido se ordenan según tipo de señalética como se muestra en la siguiente figura.
\insertimage[]{serviplott/produccion1}{width=6cm}{Clasificación de señalética.}
\vspace{5mm}

Ahora viene el procesos de cortar los postes omegas según sea necesario para colocar una señalética.

Las vallas peatonales se soldan por bloques y se dejan listas para despacho.
\insertimage[]{serviplott/produccion6}{width=9cm}{Postes omega.}
\insertimage[]{serviplott/produccion5}{width=8cm}{Vallas peatonales.}


	\subsubsection{Proceso de despacho}
	
	El despacho consiste en llevar a terreno la señalética o vallas peatonales desarmadas.  En el lugar se recibe a los técnicos generalmente un funcionario de la municipalidad de la comuna donde se colocará la valla peatonal o la señalética
Los técnicos ensambla la valla peatonal o la señalética en el lugar que se colocará.
Una vez finalizado el trabajo de colocación del producto, un funcionario municipal debe dar el visto bueno y firmar una carta de producto completado.
	\insertimage[]{serviplott/produccion4}{width=10cm}{Instalación señalética.}



	
	\subsection{Descripción de Outputs y sus Características}\label{output}
	El output del proceso productivo son las señaléticas listas para montar y las vallas peatonales. Serviplott también resive pedidos personalizados para la necesidad de cada cliente, que pueden estar en función por ejemplo de materiales de una calidad superior para soportar clima adverso.
	\begin{enumerate}[1)]
	\item Señalética : Se usa un papel de impresion (Vinil autoadhesivo) como principal componente donde queda la señalética. Es un material de plástico bondadoso y resistente. Pro su flexibilidad se amolda y adhiere fácilmente a cuaqluier superficie (plana o curva). Es ideal para ambientes interiores y exteriores.
	\insertimage[]{serviplott/senal1}{width=8cm}{Diseño señalética.}
	\insertimage[]{serviplott/senal2}{width=7cm}{Diseño señalética.}
	\item Vallas peatonales : Son construidas en acero bajo el diseño del siguiente plano : 
	\insertimage[]{serviplott/valla2}{width=8cm}{Planos vallas peatonales.}
	Y se ubican según la norma chilena de tránsito, sobre la ácera en forma paralela al eje longitudinal de la calzada y a una distancia entre 30 y 50 cm. del borde de la solera. Para que resulten eficaces e impidan que las personas desciendan de la calzada en zonas específicas, y sirven como contensión ante enventuales choques de auto.
	\item Poste omega : Poste acero laminado en caliente para sustentanción de señalética vial. Se hace bajo el diseño del siguiente plano.
	\insertimage[]{serviplott/poste1}{width=5cm}{Diseño poste omega para señalética.}
	\end{enumerate}
	\subsection{Caracterización del cliente}
    	En este punto se dividen dos tipos de clientes los cuales son internos y externos.
	\begin{enumerate}[1)]
	\item Clientes internos. 
	
	Entre los los clientes internos de los procesos industriales de Serviplott, se encuentran:


	\begin{enumerate}[a)]
	\item Propietarios y Gerentes : En general, la parte directiva de la empresa necesita conocer los estados financieros de esta, para así, poder tomar decisiones importantes sobre el negocio y que afectarán la continuidad de las operaciones.
	\item Empleados: Los empleados también requieren de la información financiera para desempeñar algunas actividades en situaciones específicas. Por ejemplo, para acuerdos de negociación colectiva en negociaciones en torno a las remuneraciones.
	\vspace{5mm}
	
	También es necesario tener en cuenta ciertos índices para dirigir sus esfuerzos a ciertas actividades con más prioridad que otras.
\newpage
	
	Otros clientes pueden ser definidos como:
	\item Clientes de servicios básicos.
	\item Clientes de contabilidad.
	\item Clientes de sala de producción, cliente de servicio de bodega.
	\item Clientes de servicio de Envío.
	\item Área de Administración de Decisiones.
	

	\end{enumerate}
		
	\item Clientes externos.	
	\begin{enumerate}[a)]	
	\item Municipalidades : Serviplott tiene como principal clientes distintas municipales tales como Santiago, Macul, Ñuñoa, entre otras. Las Municipalidades tienden a ser licitaciones por un periodo no inferior de 4 años por licitación en la cual cada comuna acepta la postulación de la empresa y los costos asociados a la misma. 

Las municipalidades de santiago y ñuñoa tienen distintos formas de comprar productos, en las figuras siguientes se muestran las ofertas económicas de ambas comunas. 

	\insertimage[]{serviplott/stgo}{width=7cm}{Oferta económica santiago.}
	\insertimage[]{serviplott/nunoa1}{width=7cm}{Oferta económica ñuñoa.}
	\insertimage[]{serviplott/nunoa2}{width=7cm}{Oferta económica ñuñoa.}
	\insertimage[]{serviplott/nunoa3}{width=7cm}{Oferta económica ñuñoa.}


    \item Empresas : Existen empresas que solicitan impresiones para colocar en sus instalaciones tales como demarcaciones de zonas de riesgo, ingreso solo de personal autorizado, zonas de silencio, zonas limpias, etc.



	\end{enumerate}
	\end{enumerate}

	
	\subsection{Análisis de control de calidad}
		El control de calidad son todos los mecanismos, acciones, herramientas realizadas para detectar la presencia de errores. Este radica su existencia primordial como una organización de servicio, para conocer las especificaciones establecidas por la ingeniería del producto y proporcionar asistencia al área de fabricación, para que la producción alcance estas especificaciones.	


      	\subsubsection{De tipo técnico}
      	
			El aseguramiento de calidad en las funciones operativas (instalaciones) que realiza la empresa, siendo un aspecto clave la fabricación de señaléticas (tiempos de entrega, calidad final del producto acorde a las especificaciones, disponibilidad de insumos en buen estado, etc). Para ello se cuenta con la modalidad de realizar su armado por los propios técnicos de la empresa.
	
	
	También es necesario controlar al final de cada jornada las maquinarias respecto a su correcto funcionamiento durante toda la jornada de otra forma será revisado.
	
	
	\subsubsection{De tipo comercial}
	
	Las funciones de publicidad y ventas se vuelven claves, por lo que también son revisadas en calidad, está dada por la retroalimentación de quejas de los clientes. También se busca alcanzar un prestigio en calidad y servicio al cliente, siendo favorecido mediante la inclusión de un servicio post-venta (Mantenimiento) que demuestre preocupación por el cumplimiento de promesas hacia el cliente, y que a su vez ayude a dar información sobre cómo mejorar servicio post-venta.
	
	
	\subsubsection{Políticas de calidad}
		
		Con respecto a calidad se establece un marco sobre el cual la empresa debe moverse. Este se define teniendo en cuenta los objetivos, valores organizacionales y las expectativas y necesidades de los clientes.
		
		
		Se establecen los siguientes temas con respecto a calidad :
		
		\begin{enumerate}[1)]
		\item Sociedad
		  \begin{enumerate}[a)]
		  \item Cumplir con todos los requisitos aplicables a la actividad de la empresa
		  \item  Mejorar continuamente la eficacia del sistema de gestión de calidad
		  \item  Cuidar el medio ambiente
		  \end{enumerate}
		\item Clientes
		  \begin{enumerate}[a)]
		  \item Ofrecer productos de calidad
		  \item Mejorar continuamente el diseño, estudio de propuestas, ejecución y mantenimiento de los productos.
		  \end{enumerate}
		\item Personas
		  \begin{enumerate}[a)]
		  \item Desarrollar un buen ambiente de trabajo.
		  \item Brindar oportunidades de capacitación
		  \end{enumerate}
		\item Proveedores
		  \begin{enumerate}[a)]
		  \item Desarrollar una relación que permanezca en el tiempo
		  \end{enumerate}
		\end{enumerate}
	
	\subsection{Análisis de comportamiento. Calculo Indicadores de Calidad}
	
	
	\subsection{Causas fundamentales de anomalías}
	De los datos relevados en los indicadores expuestos anteriormente, en conjunto con el levantamiento de procesos y análisis actual del mismo, se concluye que las principales problemáticas que se presentan actualmente están dadas por los registros tardíos y la poca optimización de los espacios para entrada y almacenamiento de los insumos.
	
	
	Serviplott no goza de una fluidez en el ingreso de camiones a la bodega, y esto genera tiempos de espera excesivos, a su vez la forma de control de inventario carece de tecnología, impidiendo una correcta trazabilidad interna de los insumos.
	

Estas problemáticas no permiten que la empresa se proyecte a asumir una gran cantidad de clientes al mismo tiempo, provocando un estancamiento de flujo de stcok, por cual es necesaria una mejora en el proceso de almacenaje.
    \subsubsection{Levantamiento de proceso}
    El respectivo levantamiento de procesos realizado en Bizagi, se muestra el proceso de almacenaje, con todas las tareas que implican el funcionamiento del sistema en sí.

	\subsubsubsection{Análisis  de la situación actual}
	A través del siguiente diagrama de causas y efectos se busca representar las
principales causas del estancamiento en el flujo de stock, abarcando demoras en
almacenaje e inventario, como también optimización de espacios en bodega.	
	\insertimage[]{serviplott/situacion_actual}{width=8cm}{D
iagrama de causas y efectos de la problemática actual.}


Como se puede apreciar en el modelo Ishikawa presentado, el problema radica en la
falta de tecnología, la poca optimización de espacios en bodega, accesos y los
funcionarios. A continuación se hará hincapié de manera detallada en cada uno de
ellos.

\subsubsubsection*{Falta de Tecnología:} Es una de las principales causantes del estancamiento en el flujo de stock, debido a la falta de un sistema moderno que abarque con los principales procesos de almacenaje, por ejemplo: la recepción del stock y la información reciente de las existencias de la bodega.

\subsubsubsection*{Poca Optimización de espacios en bodega: } Actualmente el espacio físico de la bodega es un gran problema a la hora de almacenar los insumos, puesto que en el piso de ésta no está administrado correctamente, sin la mobiliaria adecuada,
generando mermas, espacios sin uso y pérdida de tiempo.
\subsubsubsection*{Funcionamientos :} En Termofrio los funcionarios que están insertos en el proceso de almacenaje no son los suficientes, ni tampoco se encuentran capacitados en todas las tareas que implica el proceso.
\subsubsubsection*{Acceso :} Los actuales accesos del almacenaje del stock generan un estancamiento en el flujo, puesto que la ubicación del acceso principal impide fluida entrada y salida de camiones, y a su vez el tamaño no es el adecuado.

	\subsection{Sistemas de Control y Seguimiento del proceso}
	
	
	\subsection{Identificación y caracterización de las TIC de apoyo}
\section{Propuesta de PA de Mejora}
\section{Conclusión}
\section{Bibliografía}

	

