% Template:     Informe/Reporte LaTeX
% Documento:    Archivo de ejemplo
% Versión:      5.5.4 (16/09/2018)
% Codificación: UTF-8
%
% Autor: Pablo Pizarro R. @ppizarror
%        Facultad de Ciencias Físicas y Matemáticas
%        Universidad de Chile
%        pablo.pizarro@ing.uchile.cl, ppizarror.com
%
% Manual template: [http://latex.ppizarror.com/Template-Informe/]
% Licencia MIT:    [https://opensource.org/licenses/MIT/]

\section{Introducción}
Día a día el parque automotriz aumenta y es necesario mantener y tener nuevas señaléticas. Debido a esta necesidad es que nace Serviplott, una empresa la cual se encarga de vender servicios a distintos municipios los cuales requieren nuevas señaléticas y mantención de las señales viales ya existentes para así mejorar la experiencia de conducción.
\vspace{5mm}
Toda municipalidad tiene como tarea manejar las señales viales existentes en su sector. Al tener un número alto de señales en los distintos sectores es que las municipalidades optan por externalizar este servicio y se lo dejan a empresas importantes tales como Serviplott.
\vspace{5mm}
En el presente trabajo se desarrollará un completo análisis a la gestión de los procesos Industriales de Serviplott, identificando todos los problemas asociados a esta área, para luego proponer un plan de mejora en el ámbito de la Gestión de Procesos.
\vspace{5mm}
\insertimage[]{serviplott/logo2}{width=10cm}{Logo Empresa.}\newpage
\section{Objetivos}
	\subsection{Objetivos Generales}
Analizar un proceso industrial de la empresa “Serviplott Servicios Gráficos y Señalización Vial Ltda”, y proponer planes de mejora.
	\subsection{Objetivos Específicos}
	\begin{itemize}

		\item Realizar un relevamiento de información de la empresa seleccionada para 					diagnosticar el estado actual respecto a la organización, sus procesos 				y sistemas.
		\item Determinar las soluciones a las oportunidades de mejora detectadas.
		\item Formular las recomendaciones necesarias para implantar los proyectos y
		acciones de mejora.
		\item Realizar una propuesta de mejoramiento, que refleje las oportunidades de 		mejora posibles para la empresa.
	\end{itemize}
	\newpage
\section{Alcances Y Limitaciones}
	\subsection{Alcances}
	A partir de los antecedentes recolectados de “Serviplott”, se realizará un estudio de un proceso industrial de la organización que sea relevante de acuerdo a la información respecto a la organización,  sus procesos y sistemas, el cual contempla los aspectos, tales como la situación actual de la empresa, el soporte tecnológico, así como también las mejores prácticas, entre otros. Con toda esta información, en conjunto con las estrategias de la empresa, se definirá una lista de oportunidades de mejora, procesos, además de proponer planes de mejora para ese proceso en específico. 
	
	\subsection{Limitaciones}
	Existirá limitación en cuanto a la información a la que pudiese ser accedida desde el punto de vista financiero y/o estratégico por motivos de la empresa y de sus clientes, en este caso distintas municipalidades, por lo que algunas estrategias de mejora podrían no ser del todo óptimas. Es importante mencionar que el Plan de Mejoramiento entregado será considerado como una propuesta.
\newpage
\section{Antecedentes Generales de la Empresa}
	\subsection{Identificación de la Empresa}
	Fundada en 2002, Serviplott es una empresa conformada por profesionales del área de la ingeniería mecánica y la ingeniería en transporte. Estas características le dan a la empresa flexibilidad, rapidez y soporte técnico que demanda el mercado de la infraestructura vial. En Serviplott creen en una relación personalizada con sus clientes y por ello ofrecen tiempos de respuesta de acuerdo a sus particulares necesidades con altos estándares de calidad.
\vspace{5mm}

Inicialmente se encargaban de generar gráficos publicitarias y seguridad, también entregando proyectos de ingeniería en tránsito. Con el tiempo serviplott comenzó a especializarse solamente en soluciones de ingeniería en tránsito, donde se especializó en señales viales.\vspace{5mm}

A partir de ello, serviplott comienza a licitar en las distintas comunas para así ser una empresa de renombre en mantención e instalación de señaléticas viales. Hoy en día serviplott se encuentra encargado de comunas tales como Santiago, Lo Barnechea, Macul, Ñuñoa.\vspace{5mm}

Serviplott como empresa se encuentra en una constante actualización de técnicas y tecnologías capacitando a sus distintos trabajadores  en las distintas áreas. 
\vspace{5mm}
	\subsection{Misión}
	Serviplott es una empresa conformada por profesionales del área de la ingeniería mecánica y la Ingeniería en transportes. Estas características le dan a la empresa la flexibilidad, rapidez y soporte técnico que demanda el mercado de la infraestructura vial. En Servilplott creemos en una relación personalizada con nuestros clientes y por ello ofrecemos tiempos de respuesta de acuerdo a sus particulares necesidades con altos estándares de calidad.
	\subsection{Visión}
Consolidarnos como el mejor proveedor de productos y servicios viales, siendo reconocidos por el mercado como una empresa seria, profesionales y confiable orientada a la satisfacción de sus clientes y de los usuarios del servicio.

	\subsection{Valores}
	Nos importa el bienestar de nuestro entorno y colaboradores, preocupándonos de mantener niveles óptimos de seguridad y cumpliendo con las exigencias de las industrias que atendemos. Los materiales y productos utilizados en nuestras obras provienen de proveedores con trayectoria a nivel mundial. Por otro lado nos aseguramos de entregar un servicio de mantención y post venta con respuestas ágiles y eficaces a nuestros clientes.\\
\vspace{9mm}
Es nuestra responsabilidad cumplir con esta promesa y el mayor reconocimiento está en la decisión de nuestros clientes de preferir contratar nuestros servicios. Somos más que energía térmica: estamos comprometidos con el desarrollo y
trabajo bien hecho.

	\subsection{Mix de Productos}
	\subsection{Procesos Internos}
\section{Análisis de Proceso}
	\subsection{Identificación de Inputs}
	\subsection{Descripción Actividades}
	\subsection{Descripción de Outputs y sus Características}
	\subsection{Caracterización del cliente}
	\subsection{Análisis de control de calidad}
	\subsection{Análisis de comportamiento. Calculo Indicadores de Calidad}
	\subsection{Causas fundamentales de anomalías}
	\subsection{Sistemas de Control y Seguimiento del proceso}
	\subsection{Identificación y caracterización de las TIC de apoyo}
\section{Propuesta de PA de Mejora}
\section{Conclusión}
\section{Bibliografía}

	

